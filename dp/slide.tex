\documentclass{beamer}
%\documentclass[aspectratio=169]{beamer} % 使用 16:9 宽屏比例
\usepackage{ctex, hyperref}
\usepackage[T1]{fontenc}
\usepackage{listings}
\usepackage{color}


% other packages
\usepackage{latexsym,amsmath,xcolor,multicol,booktabs,calligra}
\usepackage{graphicx,pstricks,listings,stackengine}

\author{DrAlfred}
\title{2026 年寒假讲题}
\subtitle{浅谈 DP 及其优化}
\institute{上海市曹杨第二中学 maze.size()}
\date{2026年02月08日}
\usepackage{CNU}

% defs
\def\cmd#1{\texttt{\color{red}\footnotesize $\backslash$#1}}
\def\env#1{\texttt{\color{blue}\footnotesize #1}}
\definecolor{deepblue}{rgb}{0,0,0.5}
\definecolor{deepred}{rgb}{0.6,0,0}
\definecolor{deepgreen}{rgb}{0,0.5,0}
\definecolor{halfgray}{gray}{0.55}


\begin{document}

\kaishu
\begin{frame}
    \titlepage
    \begin{figure}[htpb]
        
    \end{figure}
\end{frame}

\begin{frame}
    \tableofcontents[sectionstyle=show,subsectionstyle=show/shaded/hide,subsubsectionstyle=show/shaded/hide]
\end{frame}

\section{序言}
\begin{frame}{序言}
    \begin{itemize}
        \item 大家能来到这里说明肯定对 DP 有一定的了解,本节课主要是教大家一些状态设计和优化手法。
        \item 讲者很菜。Luogu: $583610$,QQ: $229882561$,欢迎联系交流。
        \item 大家需要课件和配套资料可以访问 \url{https://github.com/CYEZOI/2026-Winter-OI} 或者 \url{https://share.alfredbao.cn/}。
        \item 接下来的 3h 祝大家 GL and HF,有问题随时提问!
    \end{itemize}
\end{frame}
\begin{frame}{如何定义 DP}
    \begin{itemize}
        \item 我讲之前,大家可以先说说自己的想法。 
    \end{itemize}   
\end{frame}
\begin{frame}{如何定义 DP}
    \begin{itemize}
        \item DP 实际上是一个分步生成解的自动机结构。我们通过分步生成解把指数级搜索空间压缩到多项式级别,找到解的某种权值经过某种运算后的结果。
        \item 这种情况下的 DP 大致有三类:最优化、计数、判定。在所有合法解不漏的前提下,它们需要满足的关键条件是:最优子结构、不计重、无后效性。
        \item 广义上来说,习惯于将大部分用递推解决问题的方法都称作 DP。
        \item 如果可以找到一种生成方式,能恰好(逐步)刻画出所有满足条件的解,并且中途为了判定符合条件以及为了辅助求出权值所记录的信息是局部的或可能性较少的,那就有机会使用 DP。
    \end{itemize}
\end{frame}

\section{常见结构的技巧}

\section{DP 状态设计手法}
\subsection{延后决策}

\subsection{减少状态杂说}

\section{常见的 DP 优化}
\subsection{决策单调性优化}

\subsection{斜率优化}

\subsection{Slope trick}

\subsection{WQS 二分}

\section{致谢}
\begin{frame}{致谢}
    \begin{itemize}
        \item 感谢曹杨二中 OI 教练组提供的本次交流机会;
        \item 感谢同学们几个小时的认真听讲;
        \item 感谢我的同学兼 XCPC 队(maze.size())友 JoeyJ 提供的优质例题。
        \item 感谢 YeahPotato 的优秀文章《dp 题方法总汇》提供的灵感。
    \end{itemize}
\end{frame}

\end{document}