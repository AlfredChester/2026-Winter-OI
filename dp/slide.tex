\documentclass{beamer}
%\documentclass[aspectratio=169]{beamer} % 使用 16:9 宽屏比例
\usepackage{ctex, hyperref}
\usepackage[T1]{fontenc}
\usepackage{listings}
\usepackage{color}


% other packages
\usepackage{latexsym,amsmath,xcolor,multicol,booktabs,calligra}
\usepackage{graphicx,pstricks,listings,stackengine}

\author{Alfred Bao}
\title{2026 年寒假讲题}
\subtitle{DP 篇}
\institute{上海市曹杨第二中学 maze.size()}
\date{2026年02月xx日}
\usepackage{CNU}

% defs
\def\cmd#1{\texttt{\color{red}\footnotesize $\backslash$#1}}
\def\env#1{\texttt{\color{blue}\footnotesize #1}}
\definecolor{deepblue}{rgb}{0,0,0.5}
\definecolor{deepred}{rgb}{0.6,0,0}
\definecolor{deepgreen}{rgb}{0,0.5,0}
\definecolor{halfgray}{gray}{0.55}


\begin{document}

\kaishu
\begin{frame}
    \titlepage
    \begin{figure}[htpb]
        
    \end{figure}
\end{frame}

\begin{frame}
    \tableofcontents[sectionstyle=show,subsectionstyle=show/shaded/hide,subsubsectionstyle=show/shaded/hide]
\end{frame}


\section{常见的 DP 模型}
\subsection{背包 DP}
\begin{frame}
    \begin{itemize}
        \item asdasd
    \end{itemize}
\end{frame}

\subsection{树形 DP}

\subsection{数位 DP}
% \subsection{先想所有题}
% \begin{frame}{先想所有题(20-40分钟)}
%     \begin{itemize}
%         \item 逐字读完每个题,尤其注意数据范围以及每行开头结尾是否有容易忽略的信息。(可以思想验证小样例。)
%         \item 每个题自由想10分钟左右。如果还没想出来,记录题目的研究对象、可能的寻找特例角度(特别是部分分)等认识。
%         \item 再想想,直到时间足够或者没进展了。
%         \item 最终,这一部分需要对每个题想出一个有效算法,(正解,或者暴力+性质)。
%     \end{itemize}
% \end{frame}

\section{DP 状态设计手法}
\subsection{延后决策}

\subsection{减少状态杂说}

\section{常见的 DP 优化}
\subsection{决策单调性优化}

\subsection{斜率优化}

\subsection{Slope trick}

\subsection{WQS 二分}

\section{致谢}
\begin{frame}
    \begin{itemize}
        \item 感谢曹杨二中 OI 教练组提供的本次交流机会
        \item 感谢同学们几个小时的认真听讲
        \item 感谢我的同学兼 ICPC 队友 JoeyJ 提供的优质例题
    \end{itemize}
\end{frame}

\end{document}