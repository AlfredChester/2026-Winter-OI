\documentclass{beamer}
%\documentclass[aspectratio=169]{beamer} % 使用 16:9 宽屏比例
\usepackage{ctex, hyperref}
\usepackage[T1]{fontenc}
\usepackage{listings}
\usepackage{color}


% other packages
\usepackage{latexsym,amsmath,xcolor,multicol,booktabs,calligra}
\usepackage{graphicx,pstricks,listings,stackengine}

\author{DrAlfred}
\title{2026 年寒假讲题}
\subtitle{浅谈 DP 及其优化}
\institute{上海市曹杨第二中学 maze.size()}
\date{2026年02月08日}
\usepackage{CNU}

% defs
\def\cmd#1{\texttt{\color{red}\footnotesize $\backslash$#1}}
\def\env#1{\texttt{\color{blue}\footnotesize #1}}
\definecolor{deepblue}{rgb}{0,0,0.5}
\definecolor{deepred}{rgb}{0.6,0,0}
\definecolor{deepgreen}{rgb}{0,0.5,0}
\definecolor{halfgray}{gray}{0.55}


\begin{document}

\kaishu
\begin{frame}
    \titlepage
    \begin{figure}[htpb]
        
    \end{figure}
\end{frame}

\begin{frame}
    \tableofcontents[sectionstyle=show,subsectionstyle=show/shaded/hide,subsubsectionstyle=show/shaded/hide]
\end{frame}

\section{序言}
\begin{frame}{序言}
    \begin{itemize}
        \item 大家能来到这里说明肯定对 DP 有一定的了解,本节课主要是教大家一些状态设计和优化手法。
        \item 讲者很菜。Luogu: $583610$,QQ: $229882561$,欢迎联系交流。
        \item 大家需要课件和配套资料可以访问 \url{https://github.com/CYEZOI/2026-Winter-OI} 或者 \url{https://share.alfredbao.cn/}。
        \item 接下来的 3h 祝大家 GL \& HF,有问题随时提问!
    \end{itemize}
\end{frame}
\begin{frame}{如何定义 DP}
    \begin{itemize}
        \item 我讲之前,大家可以先说说自己的想法。 
    \end{itemize}   
\end{frame}
\begin{frame}{如何定义 DP}
    \begin{itemize}
        \item DP 实际上是一个分步生成解的自动机结构。我们通过分步生成解把指数级搜索空间压缩到多项式级别,找到解的某种权值经过某种运算后的结果。
        \item 这种情况下的 DP 大致有三类:最优化、计数、判定。在所有合法解不漏的前提下,它们需要满足的关键条件是:最优子结构、不计重、无后效性。
        \item 广义上来说,习惯于将大部分用递推解决问题的方法都称作 DP。
        \item 如果可以找到一种生成方式,能恰好(逐步)刻画出所有满足条件的解,并且中途为了判定符合条件以及为了辅助求出权值所记录的信息是局部的或可能性较少的,那就有机会使用 DP。
    \end{itemize}
\end{frame}

\section{基于特殊结构的技巧}
\begin{frame}{括号序列}
    
\end{frame}
% \subsection{括号序列}
% \subsection{排列}
% \subsection{背包}
% \subsection{概率 \& 期望}
% \subsection{树形}
% \subsection{平面}

% \section{DP 状态设计手法}
% \subsection{延后决策}
% \subsection{减少状态杂说}

\section{常见的 DP 优化}
\begin{frame}{开始优化之前}
    先对自己进行一些灵魂拷问:
    \begin{itemize}
        \item 你找全这个结构的性质了吗?
        \item 你找对该对着 DP 的对象了吗?有没有可以解构的地方?
        \item 你设计的 DP 状态已经尽可能少了吗?
    \end{itemize}
    主要思考以下几个方面:
    \begin{itemize}
        \item 寻找可以直接从题目条件导出,或与题目条件等价的条件;
        \item 模拟(包括打表)小样例、特殊样例找到简化的方向和规律;
        \item 另外,如果解是一个过程,有时倒推一些必要的东西会比较容易。
        \item 联想一些与题目条件相关或类似的模型,尝试套用。
    \end{itemize}
    毕竟,最好的优化是不用其他优化。
\end{frame}
\begin{frame}{开始优化之前}
    \begin{block}{QOJ 16228 Sum of Three Inversions}
    给定长度 $n$,三个常数 $x, y, k$,计数下列序列三元组 $(A, B, C)$ 的个数:

    \begin{itemize}
        \item $A, B, C$ 均为长度为 $n$ 的序列;
        \item $\forall i \in [1, n], (A_i, B_i, C_i)$ 是 $(1, 2, 3)$ 的一个排列。
        \item $A$ 中恰好有 $x$ 个 $1$,$y$ 个 $2$。
        \item 序列 $A, B, C$ 中的逆序对总数为 $k$。
    \end{itemize}

    $n \le 50$,答案对给定的常数 $m$ 取模。
    \end{block}

    尝试写出所有你认识到的性质。你能做到的最优复杂度是什么?
\end{frame}
\begin{frame}{开始优化之前}
    \begin{block}{Solution}
    \begin{itemize}
        \item 可以直接对 $D_i = (A_i, B_i, C_i)$ 计数。发现大多数长度为 $3$ 的排列前后之间可以贡献的逆序对数量为 $1$,相同的排列之间贡献为 $0$,有一些排列前后之间可以贡献 $2$ 个逆序对;
        \item 可设 $f_{c_1, c_2, c_3, i}$ 表示现在分别有 $c_1, c_2, c_3$ 个组内的第一个为 $1$ 排列,第一个为 $2$ 排列,第一个为 $3$ 排列,当前逆序对贡献为 $i$ 的方案数。只有组内可能贡献 $2$ 或 $0$ 个逆序对。考虑设定基准贡献为一个逆序对,那么特殊的对之间的贡献就变成 $+1$ 或 $-1$ 了,而组间没有贡献,两组内的结构是相同的。
    \end{itemize}
    \end{block}
\end{frame}
\begin{frame}{开始优化之前}
    \begin{block}{Solution}
    \begin{itemize}
        \item 转移是容易的,然后考虑把第一组插入到第二组里面。最终答案为:

        $$
        \sum_{a = 0}^x \sum_{b = 0}^y \sum_{c = 0}^z \sum_{i = -\frac{n(n-1)}{2}}^{\frac{n(n-1)}{2}} \binom{n}{a + b + c} f_{a, b, c, i} f_{x - a, y - b, z - c, k - i}
        $$

        \item 时空复杂度为 $O(n^5)$,真能跑满吗?

            分析一下,$c_1 + c_2 + c_3 \le n$,取值方案只有 $\binom{n + 3}{3}$ 种。再加上 $-\binom{c_1 + c_2 + c_3}{2} \le i \le \binom{c_1 + c_2 + c_3}{2}$。实测状态数大概只有 $10^7$ 左右。
    \end{itemize}
    \end{block}
\end{frame}
\begin{frame}{状态数的自简化}
    
\end{frame}
% \subsection{决策单调性优化}
% \subsection{斜率优化}
% \subsection{Slope trick}
% \subsection{WQS 二分}

\section{致谢}
\begin{frame}{致谢}
    \begin{itemize}
        \item 感谢曹杨二中 OI 教练组提供的本次交流机会;
        \item 感谢同学们三个小时的认真听讲;
        \item 感谢 YeahPotato 的优秀文章《dp 题方法总汇》提供的灵感。
        \item 感谢我的同学兼 XCPC 队(maze.size())友 JoeyJ 提供的优质例题。
    \end{itemize}
\end{frame}

\end{document}